\eabstract{
    Cloud gaming, which runs games on cloud servers instead of player-owned gaming
    consoles or PCs, has been attracting much attention from the gaming industry.
    Since a player's device only receives the game screen streamed from the cloud
    server, even poorly performing mobile devices such as smartphones or tablets
    can provide a high-quality gaming experience. However, cloud gaming suffers
    from high response delay (duration between the player's input and when the
    input is reflected on the screen). The breakdown of the response delay
    includes the rendering time, video encoding/decoding time, and network delay
    between the player's device and the cloud server, but recent studies have
    shown that network delay is most significant. Therefore, reducing the network
    delay is crucial to improve the Quality of Experience (QoE) of cloud gaming.
    In this thesis, I propose a system that reduces the network latency by
    applying volunteer computing, a concept where computation is distributed
    across a large number of computing resources contributed by volunteers, and by
    running the cloud game server on a computer geographically close to the
    player. I implemented a prototype of the proposed system and evaluated the
    communication performance and the frame rate during gameplay to quantify the
    QoE. The results reveal the conditions where the proposed system provides
    better QoE than conventional cloud gaming systems and demonstrates that the
    proposed system does not degrade the QoE of gameplay.
}