\section{背景}
本章では、研究の背景および既存の研究について述べる。

\subsection{クラウドゲーミング}
クラウドゲーミングシステムはOnLiveやGaikaiといった商用サービスから始まったが、それらのクローズドシステムではゲームプレイ体験や可用性、スケーラビリティといった要素のテストベッドとして活用することは困難である。Huangら\cite{gaminganywhere}は、研究開発に利用できる高い拡張性、移植性を備えたオープンソースのクラウドゲーミングプラットフォームであるGamingAnywhereを開発した。GamingAnywhereは以下の3つの設計目標に沿って設計されている。
\begin{itemize}
    \item オープンソースのシステムであり、ビデオストリーミングなどのコンポーネントを、異なるアルゴリズムや規格、プロコトルで実装した別のコンポーネントに容易に置き換えが可能である
    \item クロスプラットフォームであり、Windows、Linux、OS Xで利用可能である
    \item メモリコピー回数を減らす等の時間的/空間的なオーバーヘッドを最小化し、効率的に動作する
\end{itemize}
GamingAnywhereはこれらの実験を通して実装された。大規模な実験を通して商用クラウドゲーミングシステムであるOnLiveとStreamMyGameを大幅に凌駕することが示された。GamingAnywhereと比較すると、OnLiveとStreamMyGameは内部における処理遅延が最大で3倍および10倍になり、またビデオ品質についてはそれぞれ3dbおよび19db低くなるという結果が示されている。また、GamingAnywhereはクラウドゲーム開発者やシステム研究者に公開され、クラウドゲーミングシステムのテストベッドとして使用することができる。

Sheaら\cite{cloudperformance}は、OnLiveを対象としてクラウドゲーミングプラットフォームのパフォーマンスを測定した。その結果、ストリーミング品質およびインタラクションの遅延が、クラウドゲーミングを構成するゲーム、計算機、ネットワークのすべての部分で課題を抱えていることを指摘した。また、携帯電話のネットワークにおいて200msを超えるネットワーク遅延の回線は珍しくなく、これにより多くのゲームにおいてインタラクション遅延が高くなりすぎる可能性について指摘した。さらに、ゲーム画面のレンダリングとゲームロジックの一部をプレイヤー端末のローカルで実行することにより、インタラクションの遅延などのクラウドゲーミングの課題の一部を隠蔽するシステムの可能性についても言及している。

クラウドゲーミングが抱える遅延の課題について、Leeら\cite{outatime}は、将来起こりうる画面フレームを投機的にレンダリングするOutatimeを提案した。Outatimeは以下の処理を組み合わせて構成されている。
\begin{itemize}
    \item 将来のゲームの状態予測
    \item 画像ベースレンダリングとゲームイベントのタイムシフトによる状態近似
    \item チェックポイントとロールバックの機構による予測の誤りが前方に伝播することの防止
    \item 帯域幅の節約のための状態圧縮
\end{itemize}
投機的にレンダリングされた画面フレームは一つのRTTに事前にプレイヤーの端末に配信され、予測と異なる状態に遷移した際に迅速に回復する。これにより、クラウドゲーミングで発生するネットワーク遅延を最大120ms隠蔽できると述べている。一方で、ゲームのステージ内をワームホールでテレポートするようなイベントについては、この手法では対応できないということも指摘している。

クラウドゲーミングで発生する遅延のうちでも大きな要因であるネットワーク遅延の課題に対し、Hongら\cite{placing}はクラウドゲームサーバとして動作するVM配置の最適化によって解決を試みた。プレイヤーのQoEとプロバイダの純利益との間のトレードオフであるクラウドゲーミングの最適化を行うためのヒューリスティックアルゴリズムを開発した。この中で、クラウドゲームサーバの役割を果たすVMを、プレイヤーの端末から最もネットワーク遅延が小さい位置にあるサーバに割り当てるという最適化を行っている。また、開発したアルゴリズムを評価するために大規模なシミュレーションを実施しており、結果としてアルゴリズムが以下のことを示している。
\begin{itemize}
    \item 最適に近いソリューションが得られる
    \item 20000台のサーバと40000人のプレイヤーを抱える大規模なクラウドゲーミングサービスへの拡張が可能である
    \item 最先端の既存ヒューリスティックアルゴリズムを上回る結果が得られること
\end{itemize}
一方、この手法においてもプレイヤー端末からデータセンターまでのネットワーク遅延を回避できないという課題の解決には至っていない。



\subsection{ボランティアコンピューティング}
High-Performance Task Distribution for Volunteer Computing 
Andersonら\cite{boinc}は、


