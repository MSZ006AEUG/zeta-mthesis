\section{まとめと今後の課題}
本研究では、ボランティアが研究する地理的に近傍の遊休コンピュータのリソースを活用することによるクラウドゲーミングシステムを提案した。既存のクラウドゲーミングアーキテクチャにおいて、プレイヤーの端末からデータセンターへ接続する際のネットワーク遅延を回避することはできない。この課題を解決するため、広域に分散したボランティアの計算機資源でクラウドゲームサーバをホストするシステムを考案した。実験を通して、提案システムはクラウドゲーミングのプレイにおいてネットワーク遅延やスループットが許容量に収まることを示した。また、ゲームプレイ中のフレームレートについて、低スループットのネットワークを使用した際でも低下が見られないことを示した。これにより、提案システムがクラウドゲーミングのプレイにおけるQoEを低下させることなくサービスを提供できることを示した。

今後の展望としては、利用可能な遊休コンピュータの中から、プレイヤーPCからのネットワーク遅延が小さいものを探索する機能を持つボランティアクラウドゲーミングコントローラの実装を検討している。また、今回の実験は遊休コンピュータとプレイヤーPCがそれぞれ1台ずつの構成で行ったが、これらの数を増やした上でのシステムの負荷試験についても検討する。この際、ゲームのジャンル毎に許容遅延などのQoEの許容範囲を考慮し、それぞれについて遊休コンピュータのマッチングを調整する必要が考えられる。
また、ゲーム毎にレンダリングに必要な時間に違いがある可能性についても考慮する必要が考えられる。高画質かつ高フレームレートのゲームなどのレンダリングやエンコードの負荷が大きいゲームの場合、総応答遅延が長くなる。これは相対的にネットワーク遅延が小さくなることを意味する。このとき、提案システムでQoEを維持するために、より近傍の遊休コンピュータを使用して総応答遅延を削減する必要が想定される。

