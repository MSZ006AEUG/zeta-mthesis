\abstract{
従来のプレイヤーがゲーム専用機やゲーミングPCを所有して行うゲームプレイに対し、クラウドサーバでゲームを動作させてゲーム画面をストリーミングするクラウドゲーミングがゲーム業界で注目されている。クラウドゲーミングはプレイヤーの端末ではストリーミングされた画面の描画とプレイヤーの入力の送信のみを行うため、スマートフォンやタブレットなどの性能が貧弱な端末でも高品質なゲーム体験を得られる。一方で、クラウドゲーミングはプレイヤーが入力を行ってからストリーミングされる画面に反映されるまでの応答遅延が課題である。この応答遅延は、クラウドゲームサーバにおけるレンダリング処理時間やストリームのエンコード・デコード処理時間を含むが、特にクラウドゲームサーバとプレイヤー端末間のネットワーク遅延の割合が大きい。クラウドゲーミングのプレイヤーのQuality of Experience (QoE) を担保するために、このネットワーク遅延を解消する必要がある。本研究では、広域分散した計算機資源を活用する枠組みであるボランティアコンピューティングの仕組みを応用し、クラウドサーバよりも地理的に近傍の計算機でクラウドゲームサーバを動作させることによってネットワーク遅延を解消するシステムを提案する。また、提案手法を評価するために、システムの実装を行い、通信性能の評価およびゲームプレイにおけるQoEを測定するためにフレームレートの評価を行った。提案システムがクラウドゲーミングのプレイにおけるQoEを低下させることなくサービスを提供できることを示した。
}