\section{実装}
\pagenumbering{arabic}

現在の提案システムの実装はサーバマシンとクライアントマシンがXMPPサーバから取得した情報をもとに直接リンクを張ってクラウドゲーミング通信を展開する方式になっている。クラウドゲームサーバ/クライアントにはオープンソースクラウドゲーミングプラットフォームであるGamingAnywhereを使用している。また、パブリッククラウド等で展開しているサービスではなく、一般のユーザコンピュータでサービスを展開する手法特有の課題とその対処についても本章で述べる。

\subsection{実装上の課題}
パブリッククラウドやビジネス向けのサーバでサービスを展開する場合と異なり、提案システムはクラウドゲームサーバをボランティアの提供するユーザコンピュータ上で展開する。通常ユーザコンピュータはNATやファイアウォールの背後にあり、また固定IPアドレスを持っていないことも多いため直接通信することが難しい。これにより提案システムが必要とする通信において、大きく2つ問題が生じる箇所が出てくる。一つはクラウド上のコントローラから遊休コンピュータへクラウドゲームのホスティング等の直接命令を送ることができないということである。もう一つは、クラウドゲームサーバ/クライアントを展開する遊休コンピュータとプレイヤーPC間での通信において、双方向的な直接通信を展開できないという点である。

\subsection{VCコントローラとエージェントの連携}
VCコントローラと遊休コンピュータ上で動作するVCホストエージェントとの通信の課題についてはgRPC\cite{grpc}を用いる。gRPCはGoogleが開発しているオープンソースのRPCサービスで、異なるコンピュータ間あるいはマイクロサービス間で情報をやり取りするのに使われる。gRPCではクライアントアプリケーションがローカルオブジェクトであるかのようにサーバアプリケーションのメソッドを直接呼び出すことができるため、分散アプリケーション等の実装に適している。サーバ側ではサービスを定義してそのインターフェースを実装する。クライアント側ではサーバと同じメソッドを提供するスタブを介してサーバアプリケーションの機能を使えるようにしているのが特徴である。

gRPCのResponse Streaming gRPCという機能は、単一のリクエストに対して複数のレスポンスを返すことが可能である。これを使用することで、VCクライアントが送信した単一のゲームプレイリクエストに対して、ACKや起動報告、ホスティング終了時の完了報告、エラーの通知など様々なレスポンスを返すことができる。

\subsection{クラウドゲームサーバ/クライアント間のP2P通信}
実際にリモートでのゲームプレイを実現するクラウドゲームサーバとクラウドゲームクライアントにはGamingAnywhereを使用する。GamingAnywhereはオープンソースのクラウドゲーミングプラットフォームであり、ユーザのコンピュータにインストールして設定を最適に変更しつつ実験を行えるクラウドゲームテストベッドである。遊休コンピュータ上にGamingAnywhereサーバ、プレイヤーPC上にGamingAnywhereクライアントを起動し、GamingAnywhereサーバが展開するRTSPサーバにクライアントが接続することでクラウドゲームのプレイが開始される。

ここで、ユーザコンピュータで動作するGamingAnywhereサーバ/クライアント間で双方向的な直接通信を行えないという問題がある。これを解決するため、GamingAnywhereの通信を行うリンクに対しEdgeVPN\cite{edgevpn}を使用する。EdgeVPNは、分散エッジソース全体にスケーラブルVPNを展開するためのオープンソースソフトウェアである。パケットキャプチャ、暗号化、トンネリング、転送およびNATトラバーサルのサポートが組み込まれている。エッジデバイスとNAT/ファイアウォール、およびクラウドコンピューティングリソースの背後にあるネットワークアドレスに透過的に接続することで、インターネットを介したトラフィックをP2Pで暗号化およびトンネリングすることができる。EdgeVPNリンクはオープンソースのWebRTCフレームワークによって実装されたSSLベースのトランスポートレイヤプロトコルセキュリティで暗号化および認証され、ノード間の通信はトンネルを介してUDPに基づくDatagram TLS(DTLS)\cite{dtlc}を使用する。また、EdgeVPNはXMPPプロトコル\cite{xmpp}を使用してピアとの接続情報を検出および交換する。パケット交換とルーティングは分散されているため、スケーラブルなP2Pオーバレイを展開しつつ、メンバーシップの一元管理も可能である。

EdgeVPNはSt Justeら\cite{tincan}が開発したTinCanに基づいている。TinCanは、XMPPサーバを使用してエンドツーエンドのVPNトンネルをブートストラップし、分離されたコントローラ/データパスモデルをサポートしている。また、ノードが接続先にパブリックIPやポートを知らせるためのリフレクションサービスとしてSTUNプロトコル\cite{stun}を使用し、制限の強いファイアウォールやSymmetric NATの背後にあり直接P2P接続を構築できないノードがある場合はリレーサービスとしてTURNプロトコル\cite{turn}を使用してトラフィックをプロキシしている。


\subsection{システム動作}

