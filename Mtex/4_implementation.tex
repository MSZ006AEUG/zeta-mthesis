\section{実装}
\pagenumbering{arabic}

現在の提案システムの実装はサーバマシンとクライアントマシンがXMPPサーバから取得した情報をもとに直接リンクを張ってクラウドゲーミング通信を展開する方式になっている。クラウドゲームサーバ/クライアントにはオープンソースクラウドゲーミングプラットフォームであるGamingAnywhereを使用している。また、パブリッククラウド等で展開しているサービスではなく、一般のユーザコンピュータでサービスを展開する手法特有の課題とその対処についても本章で述べる。

\subsection{実装上の課題}
通常ユーザーコンピュータはNAT/FWの背後にあるため、直接通信ができない
1.クラウドから遊休コンピュータへ直接命令を送れない
2.遊休コンピュータ・プレイヤーPC間で双方向的な直接通信ができない


\subsection{VCコントローラとエージェントの連携}
gRPC Response Streaming
Googleが開発しているオープンソースのRPC.異なるコンピュータで動作するサービス間で情報をやりとりするのに使われる。単一のリクエストに対して複数のレスポンスを返すことができるので、VCホストの完了報告などを受け取れる。

\subsection{クラウドゲームサーバ/クライアント間のP2P通信}

GamingAnywhere
クラウドゲームサーバとクラウドゲームクライアントとしてGamingAnywhereを使用する

EdgeVPN
P2P型のオーバーレイネットワークツール。ネットワークのユーザ/グループ管理が可能。TinCanの論文引用して紹介する。

\subsection{システム動作}

