\section{はじめに}
ゲーム産業は娯楽産業の中でも大きな収益を上げている産業であり、2019年の世界ゲームコンテンツ市場の規模は15兆6898億円と推定されている\cite{prtimes}。国内でも10年連続で成長しており、市場規模は1兆7330億円となっている。その中でも家庭用ハードや家庭用ソフトと比較して、スマートフォンのアプリやPC向けのオンラインゲームなどといったオンラインプラットフォームのゲームの市場規模が年々大きく拡大している。このようにゲームのプレイスタイルがゲーム専用のハードを購入するというものから汎用デバイスでのゲームプレイへと変化を見せている。こういった状況下で注目されている新たな手法で展開されるゲームサービスの一つにクラウドゲーミングがある。

従来のゲームプレイは、プレイヤーがゲーム専用ハードやゲーミングPC等を所有し、その上でゲームを動作させることによって実現されている。一方、クラウドゲーミングというサービスにおいては、クラウドサーバ上でゲームを動作させてその画面をクライアントであるプレイヤーの端末にストリーミングすることで、ゲームをネットワーク越しにプレイすることを可能にしている。このとき、プレイヤーがゲームプレイに使用する端末は、クラウドサーバより送信されるゲーム画面の再生とプレイヤーの操作のサーバへの送信だけを行う。この仕組みによって、スマートフォンやタブレット端末等の性能が貧弱なデバイスでも、従来は高価なゲームハードやゲーミングPCを使用しなければ体験できなかったような高品質なゲーム体験を得られることが期待されている。

クラウドゲーミングは商用のサービスも展開の展開もある。過去には英国のOnLive\cite{onlive}やアメリカのGaikaiがクラウドゲーミングサービスを展開していた。これらは既にサービスを終了しているが、ソニーが2012年にGaikaiを買収し、2015年にサービスを閉鎖したOnLiveの資産を取得した\cite{onlive-sony-gaikai}。同年に、ソニーは新たにクラウドゲーミングサービスのPlayStation Now\cite{ps-now}を開始した。また、Googleも2019年にクラウドゲーミングサービスであるGoolge Stadia\cite{stadia}を開始した。同社のウェブブラウザであるGoogle Chromeをインターフェースとしているのが特徴で、ユーザーへのメディアのストリーミングに動画配信サービスのYouTubeを使用している。現在は日本を含まない14カ国で展開されている。また、グラフィックプロセッシングユニットのメーカーとして知られるNVIDIAが提供するクラウドゲーミングサービスGeForce NOW\cite{geforce-now}は、従来PCゲームをプレイしていたプレイヤーをメインターゲットに据えており注目されている。他に、MicrosoftのProject xCloudやAmazonのLunaが海外でサービス開始されている。

商用サービスだけでなく研究用のプラットフォームもあり、みたいな感じでつなげる
(この辺でGamingAnywhereの話とかする?)

(クラウドゲーミングは遅延が課題ですという話を論文引用しながら書く)(サーベイ論文使ったらもっといろんな課題の話できるな)\cite{cloudgaming-survey}
クラウドゲーミングの課題はユーザ目線で高品質な画質の担保、十分なネットワーク帯域幅の確保、伝送データ圧縮・ストリーミング技術、画面表示や操作の遅延の最小化など。プロバイダ目線でゲーム環境の仮想化、サーバにおける負荷分散

(ボランティアコンピューティングの話はBOINCの引用でいいかな)

(研究目的を書く)
クラウドのデータセンターでゲームが動作しているとデータセンターまでの遅延を避けることは不可能。
ボランティアが提供する地理的に近傍の遊休コンピュータのリソースを利用するクラウドゲーミングフレームワークを提案。プレイヤーから見て近傍の遊休コンピュータ上でクラウドゲームサーバを動作させる。ネットワーク遅延削減によりプレイヤーが体験する遅延を減少させる





 
