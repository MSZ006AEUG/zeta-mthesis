\section{はじめに}
ゲーム産業は娯楽産業の中でも大きな収益を上げている産業であり、2019年の世界ゲームコンテンツ市場の規模は15兆6898億円と推定されている\cite{prtimes}。国内でも10年連続で成長しており、市場規模は1兆7330億円となっている。その中でも家庭用ハードや家庭用ソフトと比較して、スマートフォンのアプリやPC向けのオンラインゲームなどといったオンラインプラットフォームのゲームの市場規模が年々大きく拡大している。このようにゲームのプレイスタイルがゲーム専用のハードを購入するというものから汎用デバイスでのゲームプレイへと変化を見せている。こういった状況下で注目されている新たな手法で展開されるゲームサービスの一つにクラウドゲーミングがある。

従来のゲームプレイは、プレイヤーがゲーム専用ハードやゲーミングPC等を所有し、その上でゲームを動作させることによって実現されている。一方、クラウドゲーミングというサービスにおいては、クラウドサーバ上でゲームを動作させてその画面をクライアントであるプレイヤーの端末にストリーミングすることで、ゲームをネットワーク越しにプレイすることを可能にしている。このとき、プレイヤーがゲームプレイに使用する端末は、クラウドサーバより送信されるゲーム画面の再生とプレイヤーの操作のサーバへの送信だけを行う。この仕組みによって、スマートフォンやタブレット端末等の性能が貧弱なデバイスでも、従来は高価なゲームハードやゲーミングPCを使用しなければ体験できなかったような高品質なゲーム体験を得られることが期待されている。

クラウドゲーミングは商用のサービスの展開もある。過去には英国のOnLive\cite{onlive}やアメリカのGaikaiがクラウドゲーミングサービスを展開していた。これらは既にサービスを終了しているが、ソニーが2012年にGaikaiを買収し、2015年にサービスを閉鎖したOnLiveの資産を取得した\cite{onlive-sony-gaikai}。同年に、ソニーは新たにクラウドゲーミングサービスのPlayStation Now\cite{ps-now}を開始した。また、Googleも2019年にクラウドゲーミングサービスであるGoolge Stadia\cite{stadia}を開始した。同社のウェブブラウザであるGoogle Chromeをインターフェースとしているのが特徴で、ユーザーへのメディアのストリーミングに動画配信サービスのYouTubeを使用している。現在は日本を含まない14カ国で展開されている。また、グラフィックプロセッシングユニットのメーカーとして知られるNVIDIAが提供するクラウドゲーミングサービスGeForce NOW\cite{geforce-now}は、従来PCゲームをプレイしていたプレイヤーをメインターゲットに据えており注目されている。他に、MicrosoftのProject xCloudやAmazonのLunaが海外でサービス開始されている。

商用サービスだけでなく、研究開発用のクラウドゲーミングプラットフォームも存在する。Huangら\cite{gaminganywhere}は、既存のクローズドソースのシステムではクラウドゲーミングを体験するためのテストベッドの設置が困難であることから、オープンソースのクラウドゲーミングプラットフォームであるGamingAnywhereを開発した。GamingAnywhereはWindows、Linux、OS X上で実装されており、クライアントはiOSやAndroid等の他のOSにも移植が可能である。また、GamingAnywhereは詳細な設定を可能にしていて、更にオープンソースであるため拡張的な実装が可能であるなど、クラウドゲーミングのシステム研究のテストベッドの構築に適している。

従来のゲームシステムとは一線を画するクラウドゲーミングには、依然として重要な課題が存在している\cite{cloudgaming-survey}。クラウドゲーミングプロバイダの立場から見た課題としては、ゲーム環境の仮想化やサーバにおける負荷分散などといった課題がある。一方、システムのユーザーであるプレイヤーが知覚するゲーム体験の品質の評価指標であるQuality of Experience(QoE)
を確保することも重要な課題である。クラウドゲーミングのプレイにおけるQoEの確保に必要な課題の構成要素としては、以下のようなものがある。
\begin{itemize}
    \item ストリーミングされるゲーム画面の高品質な画質の担保
    \item ゲーム画面のリアルタイムストリーミングに耐えうる、充分なネットワーク帯域の確保
    \item 伝送データ圧縮や効率的なストリーミング技術
    \item プレイヤー端末での画面表示やプレイヤーによる操作が画面に反映されるまでの遅延の最小化
\end{itemize}
中でもプレイヤーの操作に対する応答性に直結する遅延の問題は重要である。動画配信プラットフォームにおけるライブストリーミングの場合、途切れることなく安定した動画の配信を行うためにバッファリングを行うことで対処する場合がある。しかしクラウドゲーミングは多くの場合、リアルタイムかつインタラクティブな性質を持つコンテンツであるためこの方法を使用することができない。そのため、クラウドサーバ上でのゲーム画面生成の高速化/効率化や、伝送データの圧縮、ネットワーク遅延の最小化などが課題となっている。

(ボランティアコンピューティングの話はBOINCの引用でいいかな)

(研究目的を書く)
クラウドのデータセンターでゲームが動作しているとデータセンターまでの遅延を避けることは不可能。
ボランティアが提供する地理的に近傍の遊休コンピュータのリソースを利用するクラウドゲーミングフレームワークを提案。プレイヤーから見て近傍の遊休コンピュータ上でクラウドゲームサーバを動作させる。ネットワーク遅延削減によりプレイヤーが体験する遅延を減少させる





 
